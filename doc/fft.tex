\documentclass[12pt, letterpaper, oneside]{amsart}
\usepackage{amsfonts, amsmath, amsthm, amssymb}
\usepackage[width=6.5in, height=9.0in]{geometry}
% \usepackage{kpfonts}
\usepackage{unicode-math}
\setmainfont{Lato Regular}
\setmathfont{texgyreschola-math.otf}
%\setmathfont{texgyreschola-math.otf}[range=sfit->it]
%\setmathfont[range=it]{Lato Italic}
\setmathfont{Lato Italic}[range=it]
%\setmathfont{latinmodern-math.otf}
\usepackage{url}

% Numbering and "theorems"
\theoremstyle{definition}
\newtheorem{para}{}[section]
\newtheorem{subpara}{}[para]
\newtheorem{observation}[para]{Observation}
\newtheorem{definition}[para]{Definition}

\theoremstyle{plain}
\newtheorem{theorem}[para]{Theorem}
\newtheorem{lemma}[para]{Lemma}
\newtheorem{proposition}[para]{Proposition}
\newtheorem{corollary}[para]{Corollary}
\newtheorem{conjecture}[para]{Conjecture}
\newtheorem{claim}[subpara]{}

\numberwithin{equation}{para}
\numberwithin{figure}{section}

\parindent=0pt
\parskip=10pt
\linespread{1.2}

\newcommand{\CC}{\mathbb{C}}
\newcommand{\FFT}{\text{FFT}}
\newcommand{\IFFT}{\text{IFFT}}
\renewcommand{\th}{\text{th}}

\begin{document}
\title{The FFT with multiprecision arithmetic}
\author{Marc Culler}
\date{2018}
\maketitle
\section{Introduction}

The PE module needed to be able to compute a discrete Fourier Transform with
multiprecision floating point arithmetic for its polynomial interpolation step.

A google search for Fast Fourier Transform on the arXiv.org site returns about
65,500 results.  The arXiv search engine returns 766 papers. A MathSciNet search
finds 2450 papers.  There are many software projects devoted to implementations
of the FFT, including FFTW, FFTPACK, FFTE, KISSFFT, CUFFT, ACML, ESSL, and
implementations embedded in numpy, ESSL, MKL.  Yet, as far as I could tell, none
of these packages provided an easy path to providing a multi-precision FFT for
PE.  And none of the many discussions of the mathematical background for the FFT
that I found proved to be particularly cogent.  The best discussion that I found
was some notes for a course given by Richard Fateman at
Berkeley:\break
\url{https://inst.eecs.berkeley.edu/~cs282/sp02/readings/fftnotes.pdf}.

The vast size of the literature and the enormous complexity and inflexibility of
existing implementations was intimidating.  But we eventually learned how simple
the ideas were, and wrote a short simple Python module which computes an FFT
using either Python's complex arithmetic or Sage's multiprecision complex number
class, allowing input arrays whose size has prime factors $2$ or $3$.  This
limitation on array sizes provide adequate flexibility for our purposes.  A
glance at the code in {\tt mpfft.py} shows that it would be trivial to adapt to
other choices of arithmetical systems, as one would expect from an object
oriented implementation.  The time required to compute an FFT with this code is
so small compared to the time spent in PE solving linear systems for Newton's
method that there was no need for complicated optimizations to compensate for
the speed of python.  The entire module is about 100 lines of code, including
comments.

This document is my attempt to write a simple, coherent description of the
mathematical ideas that underly the FFT algorithm.  I attempted to make the
discussion as straightforward as possible using standard mathematical
concepts, to avoid jargon and to emphasize the main ideas.

\section{Definition of the FFT}
Let $N > 1$ be an integer and fix a field $F\subset\CC$ such that the polynomial
$z^N - 1$ splits over $F$, i.e. assume that $F$ contains all of the $N^\th$
roots of $1$.  (Our assumption that $F$ is a subfield of $\CC$ fits our intended
application and simplifies this discussion, but could easily be relaxed.)

We will consider the quotient $L_N$ of the ring of polynomials in $z$ over $F$
by its principal ideal with generator $z^N - 1$:
$$L_N = F[z]/(z^N-1).$$

We shall use the symbol $\star$ to denote the multiplication operator in $L_N$
as a reminder that polynomial multiplication can also be viewed as convolution.
The ring $L_N$ has the structure of an $N$-dimensional vector space over $F$, so
$\langle L, \star\rangle$ forms an $N$-dimensional algebra over $F$.  For
notational simplicity we will start by thinking of elements of $L_N$ as (cosets
of) polynomials of degree less than $N$, although later on, in order to comply
with the standard notation, we will switch to thinking of them as Laurent
polynomials; hence the name $L$.  From this point of view, the multiplication
operation $\star$ is carried out by performing ordinary polynomial
multiplication and then reducing as much as possible using the relation
$z^N = 1$.

We will also consider the ring $$A_N = F^N$$ in which we shall denote the
coordinate-wise multiplication operator by $\circ$.  Clearly
$\langle A_N, \circ\rangle$ also forms an $N$-dimensional algebra over $F$.

\begin{lemma}\label{lemma:evaluation}
For each root $\omega$ of $z^N - 1$ there is a unique algebra
homomorphism
$$\phi_\omega:L_N \to F$$
such that $\phi_\omega(z) = \omega$.
\end{lemma}
\proof Since $L_N$ is generated as an algebra by $z$, uniqueness
follows once it is shown that $\phi_\omega$ is an algebra
homomorphism.  With the caveat that evaluation of a ``polynomial''
$f\in L_N$ is only well-defined when the value of the variable is an
$N^\th$ root of 1, we have $\phi_\omega(f) = f(\omega)$.  It is a
tautology that an evaluation map is a homomorphism whenever it is
well-defined.
\endproof

\begin{definition}\label{definition:fft}
We define $\FFT_N:L_N \to A_N$ to be the function
$$\FFT_N = \phi_1 \times \phi_\xi \times \phi_{\xi^2} \times \cdots
\times \phi_{\xi^{N-1}}$$
where $\xi$ is a primitive $N^\th$ root of 1.  It follows from
Lemma \ref{lemma:evaluation} that $\FFT_N$ is an algebra homomorphism.
\end{definition}

If $f\in L_N$ is represented by a polynomial $f(z)$ of
degree less than $N$, and if we write elements of $A_N$ as row vectors
using the usual basis, then
$$\FFT_N(f) = [ f(1), f(\xi), \ldots, f(\xi^{N-1}) ].$$
That is, $FFT_n(f)$ is the vector obtained by evaluating the polynomial $f(z)$
at the $N^\th$ roots of 1.

Next we will derive a matrix which represents the linear
transformation $\FFT_N$.  People familiar with Vandermonde matrices
will have already recognized from Definition \ref{definition:fft} that
we are simply constructing the Vandermonde matrix of the polynomial
$z^N - 1$.  But we will do the derivation from first principles.

Implicitly we have already fixed standard bases for $L_n$ and $A_n$,
namely $(z^n)_{n=0}^{N-1}$ for $L_N$ and $(e_n)_{n=0}^{N-1}$ for
$A_n$, where $e_i^j = \delta_{ij}$.  We will write vectors in $L_N$ or
$A_N$ as rows, so the linear tranformation represented by a matrix
acts by right multiplication.  The rows of the matrix $[\FFT_N]$ which
represents $\FFT_N$ are the images of the basis elements of the
domain, expressed in terms of the basis elements of the target.  Thus,
setting $\xi^N = 1$, we have
$$
[\FFT_N] = \left [
\begin{matrix}
1      & 1         & 1     & \cdots  & 1\\
1      & \xi       & \xi^2 & \cdots  & \xi^{N-1}\\
1      & \xi^2     & \xi^4 & \cdots  & \xi^{N-2}\\ 
\vdots & \vdots    &       & \ddots  & \vdots\\
1      & \xi^{N-1}  &       & \cdots  & \xi
\end{matrix}
\right ]. \eqno{(*)}
$$
In other words, $[FFT_N]_{i, j} = \xi^{(i-1)(j-1)}$.  In particular,
the complex matrix $[\FFT_N]$ is symmetric (so we would have gotten
the same matrix by using column vectors).  This makes it a very
special Vandermonde matrix.  We claim that it is orthogonal with
respect to the Hermitian inner product.

To compute the Hermitian inner product of row $j$ with row $k$, for
$j \not= k$, we let $d$ denote the greatest common divisor of $N$ and $|k-j|$
and set $M = N/d$.  Then we have
$$\big\langle\FFT_N(z^j),\; \FFT_N(z^k)\big\rangle = \sum_{n=0}^{N-1} \xi^{nj}\overline{\xi^{nk}}
= \sum_{n=0}^{N-1}\xi^{n(j-k)} = d\sum_{n=0}^M\xi^n.$$
The sum on the right hand side above is the sum of all of the roots of
$z^M - 1$, which must be $0$ because it equals the linear term of
$z^M - 1$ for $M > 1$.  This shows that the rows of $[\FFT_N]$ are orthogonal.

On the other hand, each term of
$\sum_{n=0}^{N-1} \xi^{nj}\overline{\xi^{nj}}$ is equal to $1$, which
implies that the Hermitian norm of each row of $[\FFT_N]$ is
$\sqrt{N}$.  Thus $\frac{1}{\sqrt{N}}[\FFT_N]$ is unitary, from which
we deduce that $[\FFT_N][\FFT_N]^* = NI$ and that the inverse of
$[\FFT_N]$ is $\frac{1}{N}[\FFT_N]^*$.  It is clear from the formula
$[FFT_N]_{i, j} = \xi^{(i-1)(j-1)}$ that $[FFT_N]^*_{i, j} = \xi^{-(i-1)(j-1)}$.
Letting $\IFFT_N$ denote the inverse of $\FFT_N$ we thus have
$[\IFFT_N]_{i, j} = \frac{1}{N}\xi^{-(i-1)(j-1)}$.

\section{The F in FFT}

The acronym $\FFT$ stands for ``Fast Fourier Transform''.  The second and
third letters reflect both the fact that $\FFT$ is indeed a Fourier Transform
for the cyclic group of order $N$ and that a primary domain of application
for the $\FFT$ is digital signal processing.  But here we will explain why
it is fast to compute, and how to compute it quickly.

We need two simple observations.  First, let us commit the atrocity of
reindexing our matrix entries so that rows and columns are counted
from $0$ to $N-1$ instead of from $1$ to $N$.  With this $0$-based
indexing scheme the element of $[\FFT_N]$ with index $(i,j)$ is
$\xi^{ij}$.  Next suppose that $k$ is a divisor of $N$ and consider
the $i^\th$ row of the matrix $[FFT_N]$ for $i$ a multiple of $k$.
Observe that
$$[FFT_N]_{i,j+N/k} = \xi^{ij + iN/k} = \xi^{ij} = [FFT_N]_{i,j}$$
where the second equality follows because $i$ is divisible by $k$, making
$iN/k$ a multiple of $N$.

In terms of the matrix, this means that each row having index divisible
by $k$ consists of $k$ identical blocks of length $N/k$.  A tiny bit of
inspection reveals that the blocks themselves are familiar.  Specifically

\begin{observation}\label{observation:block}
Using $0$-based indexing for the rows and columns, if $k|N$ then the
$kn^\th$ row of the matrix $[\FFT_N]$ consists of $k$ blocks of length
$N/k$, and each block is equal to the $n^\th$ row of the matrix
$[\FFT_{N/k}]$.
\end{observation}

As an immediate consequence we obtain 
\begin{proposition}
Suppose that $k|N$ and that $f\in L_{N/k}$.  Represent $f$ as a
polynomial $f(z)$ of degree at most $N/k$ and consider the element
$\widetilde f$ of $L_N$ which is represented by the polynomial
$f(z^k)$.  Then the vector $\FFT_N(\widetilde f)$ consists of $k$
blocks of length $N/k$ each of which is equal to the vector
$\FFT_{N/k}(f)$.
\end{proposition}
\proof Set $m = N/k$ and write
$f(z) = a_0 + a_1z + \cdots a_{m-1}z^{m-1}$.  When we multiply the row
vector representing $\widetilde f$ times the matrix $[\FFT_N]$, the
result is a sum of scalar multiples of rows of $[\FFT_N]$ by
coefficients of $f(z^k)$.  These coefficients are $0$ in degrees not
divisible by $k$.  So the non-zero terms are obtained by multiplying
the $kn^\th$ row of $[\FFT_N]$ by $a_n$ for $n=0, \ldots, N/k-1$.  But
the $kn^\th$ row of $[\FFT_N]$ has a block form in which each block is
equal to the $n^\th$ row of $[\FFT_{N/k}]$.  Thus the row vector
representing $\FFT_N(\widetilde f)$ has a block form in which each
block is the sum of $a_n$ times the $n^th$ row of $[\FFT_{N/k}]$.
This implies that each block is equal to $\FFT_{N/k}(f)$, as claimed.
\endproof

The second observation is even simpler.
\begin{observation}\label{observation:decomposition}
If $k|N$ then each polynomial $f(z)$ of degree less than $N$ can be
written uniquely as
$$f(z) = f_0(z^k)+ zf_1(z^k) + \cdots + z^{k-1}f_{k-1}(z^k),$$
where each $f_i(z)$ has degree less than $N/k$.  In particular, any
element $f$ of $L_N$ can be uniquely decomposed as
$$f = f_0+ Z\star f_1 + \cdots + Z^{k-1}\star f_{k-1}$$
where $Z\in L_N$ is represented by the polynomial $z$ and each $f_i$
is represented by a polynomial of the form $f_i(z^k)$, where $f_i(z)$
has degree less than $N/k$.
\end{observation}

We now have everything we need to use the ``divide and conquer''
strategy to describe an efficient recursive algorithm for computing
$\FFT_N(f)$, when $N$ is a product of powers of small primes.  For
simplicity, we will give the description for $N = 2^k$. We will
identify elements of $L_N$ with their polynomial representatives.
Then the algorithm is:

\begin{enumerate}
\item write $f(z) = f_0(z^2)+ zf_1(z^2)$ where $f_i(z)$ has degree less than $N/2$.
\item write $\FFT_N(f_i(z^2)) = [\FFT_{N/2}(f_i(z)) | \FFT_{N/2}(f_i(z))]$.
\item $\FFT_N(f) = \FFT_{N}(f_0(z^2)) + \FFT_{N}(z)\circ\FFT_{N}(f_1(z^2))$.
\end{enumerate}

The vector $\FFT_N(z)$, which is the second row of the matrix $[\FFT_N]$ can
be precomputed and $\FFT_{2^j}(z)$, for $j < k$ is obtained as a subsequence
of $\FFT_N(z)$ with stride $2^{k-j}$.  Thus the $k^\th$ stage of the recursion
essentially involves two evaluations of $\FFT_{2^{k-1}}$ and one $k$-dimensional
dot product, meaning that the running time is not much worse than $O(log N)$.

For more details about the algorithm see the actual python code in {\tt
  mpfft.py}, which is as easy to read as pseudocode would be.

\section{Conventions}

The actual values produced by the $\FFT$ depend on some choices.  If an
implementer wants to obtain the same results as existing implementations
then the same choices must be made.  The choices are
\begin{itemize}
\setlength\itemsep{1em}

\item Which primitive $N^\th$ root of $1$ to choose for $\xi$.  The standard
choice is $\xi = e^{-2\pi i/N}$; in FFTland, the circle is oriented clockwise.
This seemingly odd choice, which is also made when defining the abstract Fourier
Transform of a finite abelian group, reflects the profound fact that in the
standard definition of the Hermitian inner product it is the second factor that
gets conjugated.

\item Which ordered basis to choose for $L_N$.  The standard answer is to use
powers of $z$ as the basis elements, but to use both positive and negative
powers.  So the elements of $L_n$ are represented as Laurent polynomials
rather than standard polynomials.  Presumably this is because digital filtering
involves taking the transform of rational functions.  The powers are ordered
so that non-negative powers appear before negative powers and so that the
nonnegative values increase while the negative values decrease. The number
of non-negative powers is $\lfloor \frac{N}{2}\rfloor$.  Thus the standard ordered
basis is
$$(1, z, z^2, \ldots, z^{\lfloor \frac{N}{2} - 1\rfloor}, z^{-1}, z^{-2},
\ldots, z^{-\lfloor \frac{N}{2}\rfloor}).$$
\end{itemize}

\end{document}